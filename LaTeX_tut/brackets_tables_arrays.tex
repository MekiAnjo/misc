\documentclass[11pt]{article}
\usepackage{amsfonts, amssymb, amsmath}
\usepackage{float}
\parindent 0px  %Paragraphen werden nicht indented
\pagestyle{empty}

\begin{document}

The distributive property states that $a(b+c)=ab+ac$, for all $a,b,c\in \mathbb{R}$.\\[6pt]
The equivalence class of $a$ is $[a]$.\\[6pt]
The set $A$ is defined to be $\{1,2,3\}$.\\[6pt]
The movie ticket costs $\$11.50$.

$$2\left(\frac{1}{x^2-1}\right)$$
$$2\left[\frac{1}{x^2-1}\right]$$
$$2\left\{\frac{1}{x^2-1}\right\}$$
$$2\left \langle   \frac{1}{x^2-1}     \right\rangle$$
$$2\left |   \frac{1}{x^2-1}     \right|$$

$$\left.\frac{dy}{dx}\right|_{x=1}$$

$$\left(\frac{1}{1+\left(\frac{1}{1+x}\right)}\right)$$

Tables:\\

\begin{tabular}{|c||c|c|c|c|c|}
    \hline
    x      & 1  & 2  & 3  & 4  & 5  \\ \hline
    $f(x)$ & 10 & 11 & 12 & 13 & 14 \\ \hline
\end{tabular}

\vspace{1cm}

\begin{table}[H]
    \centering
    \def\arraystretch{1.45}
    \begin{tabular}{|c||c|c|c|c|c|}
        \hline
        x      & 1             & 2  & 3  & 4  & 5  \\ \hline
        $f(x)$ & $\frac{1}{2}$ & 11 & 12 & 13 & 14 \\ \hline
    \end{tabular}
    \caption{These values represent the function $f(x)$}
\end{table}

\begin{table}[H]
    \centering
    \def\arraystretch{1.45}
    \caption{The relationship between $f$ and $f'$.}
    \begin{tabular}{|l|p{3in}|} %p{Breite von Paragraph in inch oder cm oder pt etc.}
        \hline
        $f(x)$ & $f'(x)$                                                                                                                                     \\ \hline
        $x>0$  & The function $f(x)$ is increasing. The function $f(x)$ is increasing. The function $f(x)$ is increasing. The function $f(x)$ is increasing. \\ \hline
    \end{tabular}
    \caption{These values represent the function $f(x)$}
\end{table}

Arrays:
\begin{align}
    5x^2\, \text{place your words here} %\, in math mode macht ein Leerzeichen
\end{align}

\begin{align}
    5x^2-9=x+3 \\
    5x^2-x-12=0
\end{align}

\begin{align*}  % * macht das die Gleichungen nicht nummeriert werden
    5x^2-9    & =x+3       \\
    5x^2-x-12 & =0         \\
              & =12+x-5x^2
\end{align*}

\end{document}