\documentclass[11pt, letterpaper, twoside]{article}
\pagestyle{empty}   %turns of page numbers
\usepackage[utf8]{inputenc}
\usepackage[ngerman]{babel}
\usepackage{amsmath}
\title{Mathematical Notations in \LaTeX}
\author{Tony Khoa Nam Huynh}
\begin{document}
\maketitle
Hello! This is my first \LaTeX\ document.

A rectangle has side lengths of $(x+1)$ and $(x+3)$.
The equation $${A(x)=x^2+4x+3}$$ gives the area of the rectangle.

superscripts $$2x^3$$
$$2x^{34}$$
$$2x^{3x+4}$$
$$2x^{3x^4+5}$$

subscripts
$$x_1$$
$$x_{12}$$
$$x_{1_2}$$
$$x_{1_{2_3}}$$
$$a_0,a_1,a_2,\ldots, a_{100}$$

Greek letters
$$\pi$$
$$\Pi$$
$$\alpha$$
$$A=\pi r^2$$

Trig functions
$$y=\sin x$$
$$y=\cos x$$
$$y=\csc \theta$$
$$y=\sin^{-1} x$$
$$y=\arcsin x$$

Log functions
$$y=\log x$$
$$y=\log_5 x$$
$$y=\ln x$$

roots

$$\sqrt{2}$$
$$\sqrt[3]{2}$$
$$\sqrt{x^2+y^2}$$
$$\sqrt{    1+\sqrt{x}  }$$

Fractions
$$\frac{2}{3}$$ %\frac{Zähler}{Nenner}
About $\frac{2}{3}$ of the glass is full.\\[16pt]
About $\displaystyle \frac{2}{3}$ of the glass is full.\\[6pt]
About $\dfrac{2}{3}$ of the glass is full.\\[6pt]

$$\frac{\sqrt{x+1}}{\sqrt{x+2}}$$

$$\frac{1}{1+\frac{1}{x}}$$

\end{document}